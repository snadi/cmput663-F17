% This syllabus template was created by:
% Brian R. Hall
% Assistant Professor, Champlain College
% www.brianrhall.net

% Document settings
\documentclass[12pt]{article}
\usepackage[margin=1in]{geometry}
\usepackage[pdftex]{graphicx}
\usepackage{multirow}
\usepackage{xcolor}
\usepackage{setspace}
\usepackage{tabularx}
\usepackage{longtable}
\usepackage[obeyspaces]{url}% http://ctan.org/pkg/url
\usepackage{hyperref}% http://ctan.org/pkg/hyperref
\pagestyle{plain}
\setlength\parindent{0pt}

\newcommand{\newsec}[1]{\\ \textbf{#1:}}

\begin{document}


\begin{tabularx}{\textwidth}{Xr}
\multirow{2}{*}{\includegraphics[scale=0.25]{UA-COLOUR.png}}& \large COMPUT 663\\
&\large Fall 2017\\
\end{tabularx}

\vspace{0.2cm}
{\color{black}\rule{\textwidth}{.25ex}}%
\vspace{0.2cm}

\begin{longtable}{p{0.2\textwidth}p{0.8\textwidth}}

%Course
\newsec{Course Title} &Software Maintenance and Reuse\\

%Instructor
\newsec{Instructor}&Sarah Nadi\\
&Office: ATH 4-41\\
&Email: nadi@ualberta.ca\\
&Website: \url{http://www.sarahnadi.org}\\

%Office Hours
\newsec{Office Hours}&By appointment\\

%Lectures
\newsec{Lectures}&TR 3:30-4:50, CSC B41\\

%Moodle
\newsec{Moodle eClass}&\url{https://eclass.srv.ualberta.ca/course/view.php?id=39281} (Course schedule, readings, assignments etc. will all be available in eClass)\\

%Course description
\newsec{Course Description}&Software has become an essential component in all our daily services and products. Good software engineering practices are essential to develop resuable and reliable software. Researchers and practitioners are constantly developing innovative techniques and tooling to improve software maintenance and reuse. Goals include earlier detection of bugs, faster and more correct use of Application Programming Interfaces (APIs), and smoother refactoring of code for better reuse. 

This seminar course explores seminal and state-of-the-art research on software maintenance, evolution, and reuse. While a variety of problems related to maintenance and reuse are explored in this course, the common methodology used for studying these problems is mining software repositories. \textit{Mining software repositories} involves mining and analyzing software engineering data from various repositories such as version control repositories, bug repositories, developer forums, etc. 

The course instructor will provide some background lectures during the first two weeks of the course. During the remainder of the course, students will present and discuss papers related to the course topics.

\newsec{Course Topics} & Some of the topics covered in this course include:
\begin{itemize}
\item Software evolution
\item Bug prediction and bug detection
\item Code duplication
\item Clone detection
\item Software merging
\item API usage patterns
\item Highly configurable systems and software product lines
\end{itemize}
\\

%Prerequisites
\newsec{Prerequisites} &There are no formal prerequisites for this course, but some experience with developing and/or maintaining software is preferred (e.g., having taken a previous systems or software engineering course with a project).\\

\newsec{Course Objectives}&
\begin{itemize}
\item Understand the various software engineering (SE) data sources and how to mine them.
\item Get hands-on training on SE mining and analysis tools.
\item Become familiar with how to design quantitative and qualitative empirical SE studies.
\item Learn how to critique and write research papers.
\item Practice how to give research talks
\end{itemize}\\

%textbooks
\newsec{Text Books} &There is no required text book for this course. Various related resources and readings will be provided by the instructor throughout the course.\\

%evaluation
\newsec{Course work \& Evaluation}& Students will be graded as follows
\begin{itemize}
\item Paper Presentations (20\%)
\item Assignments (20\%)
\item Paper Critiques (10\%)
\item Participation (10\%)
\item Project (40\%)
\end{itemize}\\

%decorum
\newsec{Decorum}&
``The University of Alberta is committed to the highest standards of academic integrity and honesty. Students are expected to be familiar with these standards regarding academic honesty and to uphold the policies of the University in this respect. Students are particularly urged to familiarize themselves with the provisions of the \textit{Code of Student Behaviour }(online at \url{http://www.governance.ualberta.ca}) and avoid any behaviour which could potentially result in suspicions of cheating, plagiarism, misrepresentation of facts and/or participation in an offence. Academic dishonesty is a serious offence and can result in suspension or expulsion from the University.'' [Calendar 23.4(2)c]

\vspace{12pt}

``Audio or video recording, digital or otherwise, of lectures, labs, seminars or any other teaching environment by students is allowed only with the prior written consent of the instructor or as a part of an approved accommodation plan. Student or instructor content, digital or otherwise, created and/or used within the context of the course is to be used solely for personal study, and is not to be used or distributed for any other purpose without prior written consent from the content author(s).'' [Calendar 23.4(2)e]

\vspace{12pt}

Policy about course outlines can be found in \S23.4(2) of the University Calendar.\\

%Resources
\newsec{Resources}& Students who require accomodation in this course due to a disability are advised to discuss their needs with Specialized Support \& Disability Services (2-800 Students' Union Building see \url{http://www.ssds.ualberta.ca/} for more information)

\vspace{12pt}

For assistance with writing, please check the Centre for Writers' website (\url{http://c4w.ualberta.ca/}) that provides several forms of free writing support to students, including one-on-one feedback.

\vspace{12pt}

For more information about academic integrity, consult the office of Student Conduct and Accountability website (\url{http://www.osja.ualberta.ca/})\\

\newsec{Disclaimer} &Any typographical errors in this Course Outline are subject to change and will be announced in class.
\end{longtable}

\end{document}



